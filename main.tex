% compiling and viewing latex in os x
% brew cask install mactex
% /Library/TeX/texbin/pdflatex main.tex 

% \documentclass[twocolumn,notitlepage]{article}
\documentclass{article}

\usepackage[T1]{fontenc}
\usepackage[utf8]{inputenc}
% \usepackage{lmodern}

\usepackage{listings}
\usepackage{float}
\title{GPUCoin: A Peer-to-Peer decentralized 
% censorship resistant
% open source 
zero-knowledge 
GPU accelerated 
 zk-snarks Proof of Compute distributed computing, AI, live-streaming \& 
 mining Blockchain protocol 
 % with zk-snarks Proof of Compute \& Proof of Streaming
 % with zk-snarks \textbf{P}roof of \textbf{C}ompute \& \textbf{P}roof of \textbf{S}treaming
 % \& marketplace
 Technical Whitepaper}
\author{GPUCoin Satoshi Streaming Protocol Team}
% \date{July 26 2017}
\date{\today}
\setlength{\parskip}{1em}
\usepackage{natbib}
\usepackage{graphicx}
\usepackage{amssymb}
\usepackage{amsmath}
\usepackage{nameref}
\usepackage[normalem]{ulem}
% \usepackage{soul}
\usepackage[table]{xcolor}
\usepackage{tabularx}
\usepackage{adjustbox}
\usepackage[english, status=draft]{fixme}
\fxusetheme{color}
\renewcommand\tabularxcolumn[1]{m{#1}}% for vertical centering text in X column

\begin{document}
% \frenchspacing

\maketitle

\begin{abstract}
The \textsc{GPUCoin} team is building the next generation technology for zero-knowledge Peer-to-Peer live streaming \& distributed computing services based on block-chain technology \& a new innovative open source decentralized GPU accelerated distributed computing \& live-streaming protocol that would eliminate expensive content delivery networks. The zero-knowledge live-streaming is incentivized by GPUCoin mining validated by \textbf{P}roof \textbf{o}f \textbf{S}treaming(PoS) using zk-snarks. The zero-knowledge computing is incentivized by GPUCoin mining validated by \textbf{P}roof \textbf{o}f \textbf{C}ompute(PoC) using zk-snarks. This platform will provide a web scale cryptographically secure, fault-tolerant, self-amending distributed computing environment creating the world's first \textbf{I}nter-\textbf{P}lanetary \textbf{C}ompute \textbf{N}etwork(\textsc{IPCN}), a peer-to-peer hyper-stream protocol that makes live-streaming \& distributed computing on GPUs secure, faster, safer \& open. The \textbf{R}eal-\textbf{T}ime \textbf{S}atoshi \textbf{S}treaming \textbf{P}rotocol(RTSSP) will use peer-to-peer nodes for delivering not just live-streaming video but also archived videos verified by \textbf{P}roof \textbf{o}f \textbf{D}elivery(PoD). We strive to build the IPCN network \& GPUCoin crypto-currency token so it is truly useful to the hardcore crypto-currency \& GPU enthusiasts not just miners \& computer scientists who need GPU accelerated compute resources on the fly. We believe that solving these fundamental computer science problems as we strive to make GPUCoin a foundational cryptocurrency token will drive the entire crypto-currency, AI, machine learning \& GPU accelerated computing field forward, there by increasing the GDP of the AI, GPU accelerated distributed computing \& crypto-currency markets.

\iffalse
\sout{We will use crypto-financing (Initial Coin Offering) for capital rather than traditional venture capital \& shareholders.}
\fi

\end{abstract}
\newpage

\tableofcontents
\newpage

\section{Introduction}
\subsection{GPU accelerated computing review}
Before discussing Proof of Compute \& Proof of Streaming, GPUCoin's innovative technique for distributing coins, let’s review GPU accelerated computing at a high level.
\textbf{G}raphics \textbf{P}rocessing \textbf{U}nits (GPUs) can significantly accelerate the training process for many deep learning models. For example, GPUs can accelerate the training process for deep learning models designed for image classification, video analysis, \& natural language processing because the training process for those models involves the compute-intensive task of matrix multiplication and other operations that can take advantage of a GPU's massively parallel architecture. This architecture is well-suited for algorithms designed to address embarrassingly parallel workloads across more than a dozen areas of computer science \& other fields, including live streaming, VR/AR computer generated rendering, speech recognition, computer vision, robotics, information retrieval, natural language processing, geographic information extraction, \& computational drug discovery.

GPUCoin IPCN takes computations and maps them onto a wide variety of different hardware platforms, ranging from running inference on mobile device platforms such as Android \& iOS to modest-sized training \& inference systems using single machines containing one or many GPU cards to large-scale training systems running on hundreds of specialized machines with thousands of GPUs. Having a single system that can span such a broad range of platforms significantly simplifies the real-world use of GPUs, as we have found that having separate systems for large-scale GPU training \& small-scale deployment leads to significant maintenance burdens \& leaky abstractions.
% Just as TensorFlow/Keras provides a common interface for simplifying distributed AI \& machine learning, creating a thriving AI ecosystem, GPUCoin protocol provides a common interface for simplifying distributed GPU accelerated compute, thereby creating a thriving and profitable GPU cloud ecosystem backed by solid GPUCoin crypto-economics.

Training a deep learning model that involves intensive compute tasks on extremely large datasets can take days to run on a single processor. However, if you design your program to offload those tasks to one or more GPUs, you can reduce training time to hours instead of days. GPUCoin IPCN network is designed to address such embarrassingly parallel workloads and use GPUCoin GPC token as the currency of exchange \& transaction fees for these workloads using block-chain smart contracts. 

Over the past several years, advancements in Graphical Processing Unit (GPU) technology and GPU Accelerated Computing have played a significant and growing role in devices such as mobile phones, personal computers, and workstations, as well as in applications ranging from Artificial Intelligence/Machine Learning to 3D Video rendering to oil exploration to various automotive applications.  Every day, GPUs are being applied to a greater range of tasks and applications that can benefit from the chips’ ability to parallel process large blocks of data more efficiently.  GPUCoin is the instantiation of a new and innovative decentralized GPU peer-to-peer network to exploit the power unleashed by GPUs, on applications that could immediately benefit from the performance that such a network could bring to bear.  

Today’s live streaming video systems are complex to manage and difficult to scale, generally requiring proprietary broadcast software to encode captured video, high cost broadcast server software to push video fragments to expensive content delivery networks (CDNs) which cannot deliver the content in real-time.  When applying similar architectures to address the needs of Virtual Reality and Video360 applications, the high latency makes these virtually unusable.  CDNs suffer from having been conceived in the Web 1.0 wave of Web acceleration advancements, now 15-20 years old.  New GPU accelerated video encoding and decoding techniques have made significant strides in more efficiently handling the processing of video and other vector based graphics.  



\section{Bitcoin, Ethereum, Blockstack, Toshi, Status \& the decentralized web revolution}
 The internet is in the middle of the august beginnings of decentralization revolution: centralized proprietary services are being replaced with \textbf{decentralized open ones with open source code}; trusted parties replaced with \textbf{verifiable mathematical computation}; brittle location addresses replaced with \textbf{resilient content addresses}; inefficient monolithic services replaced with \textbf{peer-to-peer algorithmic markets} ; complex inefficient unverifiable data-structures with \textbf{verifiable efficient data-structures, Merkle Trees}. Bitcoin, Ethereum, \& other block-chain networks have proven the utility of decentralized transaction ledgers. These public ledgers process sophisticated smart contract applications \& transact crypto-assets worth tens of billions of dollars every day. These systems are the first instances of internet wide open services, where participants form a decentralized network providing useful services for pay, with no central management or trusted parties.
Virtually none of the ideas underpinning Bitcoin are new. They can all be traced to the academic literature going back decades.

 Cryptographic signatures \& public-key cryptography, cryptographic hash functions, cryptographic proof-of-work, time-stamping, Merkle trees, chains of transactions blocks, Byzantine fault tolerance, smart contracts - all of these ideas were old when Bitcoin was invented.
 Satoshi Nakamoto's achievement lays in the complex, ingenious way in which he (or she, or they) combined these ideas into a new distributed algorithm\footnote{http://queue.acm.org/detail.cfm?id=3136559}.
 
\textbf{Netscape moment}: \emph{Cambrian} explosion of crypto-currency Ðapps
 
% \newcolumntype{Y}{>{\centering\arraybackslash}X}
\centering

 \begin{adjustbox}{width=1\textwidth}
\begin{tabularx} {\textwidth}{|X|X|X|X|}
    \hline
&    \includegraphics[scale=0.3]{static/decentnew} &     \includegraphics[scale=0.3]{static/hootcoin} & \\
    \hline
\textbf{Phase} & \textbf{Internet} & \textbf{Crypto-currency} & \textbf{Reach}\\
\hline
Protocol & TCP/IP, SMTP & Bitcoin, bitgold, Ethereum & 1M People \\
\hline
Infrastructure & ISPs, lay fiber & Exchanges, secure storage & 10M people \\
\hline
Consumer Interface & Browser & User Controlled Wallet & 100M people \\
\hline
Decentralized Ðapps &  Web 2.0  & Finance 2.0 Ðapps & 1B people\\
\hline
\end{tabularx}
\end{adjustbox}

Firefox, Chrome \& IE have dominated the centralized web. New decentralized technology platforms such as Blockstack are ushering in the auspicious beginnings of decentralized web. 
We are now entering a new era of decentralized applications, block-chain technologies collectively known as Web 3.0. In the centralized web, the users are the product, their interests, preferences are managed
 % sliced and diced
 by companies such as Facebook, Twitter \& Google and sold to advertisers, enriching their small group of shareholders driven by profit, with significant barriers to entry. In the
decentralized web, the user information is private \& value creation is not about advertisements \& shareholder enrichment only. Decentralized
technologies such as Status, Toshi \& GPUCoin empower the network token holders, who can have many motivations 
\iffalse %direwolff edit
other than mere profit,
\fi
 including privacy, altruism \& a more inclusive distribution of control \& information. The emergence of Bitcoin \& subsequent block-chain technologies has generated a new digital asset class in which scarcity is based on mathematical properties \& equations re-balancing variables to maximize economic inclusiveness \& activity. Through cryptographic verification and game-theory based equilibrium, block-chain-based digital assets can be created, issued, and transmitted using software. Ownership of these cryptographic digital assets can be easily verified using public key cryptography and transfer of ownership maintained in an immutable decentralized distributed database ledger known as the block-chain. This lays the foundation for democratic transfer of value among entities in the decentralized web.
We are in the very early big-bang stages of the crypto-currency decentralized web revolution on block-chain and several miracles are happening everyday harkening to the early merry days of web 1.0.

\section{GPUCoin Crypto-currency \& Token Issuance}

The GPUCoin IPCN network includes its own built-in crypto-currency, GPUCoins GPC, which serves the dual purpose of providing a primary liquidity layer to allow for efficient exchange between various types of digital assets \&, more importantly, of providing a mechanism for paying transaction fees.

The issuance model will be as follows:

\begin{itemize}

\item GPUCoins will be released in a crypto-currency sale at the price of 1000-2000 GPUCoins GPC per BTC,
% \iffalse %direwolff edit
 a mechanism intended to fund the GPUCoin organization and pay for development that has been used with success by other platforms such as Mastercoin, Filecoin, Ethereum, Tezos and NXT.
% \fi
Earlier buyers will benefit from larger discounts. The BTC, ETH, XMR, and LTC received from the sale will be used entirely to pay salaries and bounties to developers and invested into various for-profit and non-profit projects in the GPUCoin crypto-currency ecosystem.

\end{itemize}

\subsection{GPUCoin math and token economics}
Running an AWS p2.xlarge, the lowest priced GPU unit, costs about \emph{awsdc} = \$0.9/hr. This unit is equipped with a Tesla K80 GPU, along with 64GB RAM and a 4 core CPU. 
Hence the dollar cost of an average AWS\footnote{as AWS and GCP have comparable costs} GPU instance per hour at about 1\$ per hour is a fair assumption \footnote{ varies between \~ 0.9 - 1.0 \$ in both Google compute platform and Amazon web services}

The \textbf{monthly dollar cost of AWS GPU instance} is

\emph{p} = \emph{awsdc}*24*30 = 1*24*30 = \textbf{720\$  } [1\$ per hour per aws GPU instance * 24 hours per day * 30 days per month]

Ethereum mining revenue from one AWS GPU instance

\emph{e} = 31\$

The GPUCoin instance cost is pegged at 1/5th AWS costs to stay competitive and as GPUCoin nodes do not pay infrastructure and electricity costs in the decentralized model. This can be generalized as \emph{peg} variable that GPUCoin miners can tweak based on the availability and scarcity of GPUs and render farms in their region.

\emph{peg} = 0.5

The dollar cost for a GPUCoin instance comparable to an AWS gpu instance 

\emph{gpcdc} = \emph{peg}*\emph{awsdc} =  1/5* AWS cost = 1/5 * 1\$ = 0.2\$.

Hence 0.2\$ is the dollar cost per comparable GPUCoin instance per hour as the GPUCoin IPCN network does not have to pay for infrastructure and electricity, these are sunk costs for the GPUCoin network miners.

The \textbf{monthly dollar cost of a comparable GPUCoin instance} is 

\emph{q} = \emph{gpcdc}*24*30 = 0.2*24*30 = \textbf{144\$ }

[ 0.2\$ per hour per GPU instance * 24 hours per day * 30 days per month ].


Taking \emph{t} = 20\$ for transaction fees and margins

GPUCoin mining revenue per month for 1 GPUCoin instance
\emph{r} = \emph{q} - \emph{t} = 144-20 = 124\$

\emph{r} = 4* \emph{e} \footnote{as Ethereum mining revenue from one AWS GPU instance \emph{e} = 31\$}



Hence GPUCoin mining is 4x more profitable than ethereum mining which is a profitable gpu mined cryptocurrency. This is a reasonable profit for something that takes no effort, and comparable to Bitcoin mining profits during the golden GPU mining era circa 2009-2011.

Initial GPUCoin token difficulty for mining follows the bitcoin mining model i.e., the token max limit is a hard cap of \emph{c} = 21 million GPUCoins to honor satoshi nakamoto’s invaluable bitcoin contribution to the fellow crypto brethren. Early decentralized GPUCoin nodes earn more GPUCoins as a block reward as there are fewer nodes in the network; the proof of compute and proof of streaming difficulty increases with the number of GPUCoin nodes in the network.

* whereas earlier \emph{1000 GPUCoins} would be needed to buy time in \emph{1 single GPUCoin instance}
* later as the network strengthens and more nodes join the GPUCoin network 1 single GPUCoin would be able to buy time in \emph{1000+ GPUCoin nodes} few years down the line as GPUCoin tokens appreciate in value due to scarcity of mining, a hard cap of \emph{c} = \textbf{21 million} GPUCoins and strong sustainable demand for GPUCoin instances that will be generated by the significantly cheaper than AWS pricing of comparable GPU instance.

When there are just 10 GPUCoin nodes in the network the PoC and proof of streaming difficulty is roughly 1 hour of streaming needed to earn 1 GPUCoin as block reward. As more GPUCoin nodes join the GPUCoin network the PoS difficulty goes up, when there are 10000 GPUCoin nodes in the network the mining difficulty is about 1000 hours of streaming needed to earn 1 GPUCoin as block reward.

Irrespective of the GPUCoin price and mining difficulty the GPUCoin instance dollars costs are pegged at 1/5th of the AWS instance dollar costs. This ensures that the GPUCoin dollar costs stay competitive and attractive compared to the AWS instance dollar costs in order to drive sustained GPUCoin instance demand.

As mining difficulty is adjusted based on the number of the nodes compute capacity that the network controls and the maximum number of GPUCoins that can ever be mined is hard-capped at 21 million GPUCoins the GPUCoin cryptocurrency is designed to be a deflationary currency relative to fiat currencies similar to bitcoin and litecoin.

While the miner who finds the winning block earns the GPUCoin block reward all the competing GPUCoin streaming nodes earn the transaction fees so in Bitcoin there is one winner and losers in GPUCoin there is one winner and no losers or multiple miners sharing mining rewards corresponding to their \emph{Proof of Stake} or the number of GPUCoin nodes they used for PoC or PoS mining.


\subsection{GPUCoin carbon footprint, mining, scarcity and profitability}
Bitcoin has made crypto-currencies popular and brought it to the mainstream, but it has a dark side, its ever increasing carbon footprint. In late 2013, 8.25 megatons (8,250,000 tonnes) of CO$_2$ per year was estimated to be the carbon footprint of Bitcoin per year\footnote{https://pando.com/2013/12/16/bitcoin-has-a-dark-side-its-carbon-footprint/}. In August 2017, One Bitcoin transaction uses enough energy to power 5.58 US households for 1 day and the Bitcoin network consumes 30 times more energy than the VISA network \footnote{https://digiconomist.net/bitcoin-energy-consumption}. These computers are consuming so much electricity that it’s already unprofitable to mine in some regions of the world. Since excess bandwidth and compute capacity is utilized towards streaming, encoding, object recognition and security of video and audio streams the resources otherwise would be utilized are profitably used. Since GPUCoin tokens are fairly distributed/mined to miners corresponding to their compute and bandwidth availability irrespective of how much CPU they control, the GPUCoin carbon footprint will be exponentially lower than the Bitcoin network which depends on continuously increasing complexity of the hashing required for mining. Since GPUCoin tokens can only be mined or acquired from the platform and since they are capped at a hard cap of 21 million GPUcoins, they will tend to be a scarce and valuable deflationary cryptocurrency token.

\subsection{The GPUCoin token}
The GPUCoin Token (GPC) is a native Ethereum divisible digital token with up to 18 decimal places. The total number of GPUCoin tokens to be issued is 100,000,000. For details of the distribution of these tokens, see Section \ref{sub:token_allocation_and_distribution} \nameref{sub:token_allocation_and_distribution}.

\subsection{Uses of GPUCoin token} % (fold)
\label{sub:uses_of_hoot_token}
GPUCoin Token can be used for using the IPCN platform to get bandwidth and GPU accelerated time on the network.
\iffalse
\fxerror{need to add economics of bandwidth sharing}
\fi
% subsection uses_of_hoot_token (end)
% subsection subsection_name (end)

\subsection{Insuring inflation rate does not out-pace growth of underlying economy}
The sum total of all the GPUCoins GPC coins minted at each interval can be guaranteed to be less than the calculated rate of economic growth similar to the variable bitcoin hash-cash difficulty innovation, such that current token holders can be better assured that prices are not likely to inflate and that their tokens will fall below their original value because of an over inflation of excess GPUCoin Token supply.

A further percentage of the growth can be retained as explained above to provide a price floor as an assurance of value maintenance to GPUCoin Token holders. 

\subsection{ERC223 Compatibility}
We are monitoring the ERC223 token standard proposal\footnote{https://github.com/ethereum/EIPs/issues/223} and are factoring future compatibility into the design of our Namespace GPUCoin tokens.

\subsection{Choice of Blockchain}
The seed protocol will run on top of the ethereum blockchain protocol which makes the decentralized app abstraction. We take advantage of ethereum blockchain technology to ensure a fair democratic usage of the network for all participants miners, consumers and users of GPC GPUCoin tokens.

% \fxerror*{need to edit}

%\sout{from tezos need to edit}

GPUCoin IPCN can instantiate any blockchain based protocol. Its seed protocol specifies a procedure for stakeholders to approve amendments to the protocol, \emph{including} amendments to the amendment procedure itself. Upgrades to GPUCoin IPCN network are staged through a testing environment to allow stake-holders to recall potentially problematic amendments. We believe that proof of stake blockchains are lighter alternatives to proof of work blockchains such as Bitcoin and ethereum, as the proof of work blockchains tend to have exponentially increasing CPU mining requirements as the number of participants keep growing and bringing on more GPU accelerated compute to the IPCN network.

\section{Motivation - design goals}
In our development of Hoot, we aspire to address four problems with live streaming and compute over blockchain:
\begin{itemize}
\item[-]Entertainment systems are designed to benefit the select few in Hollywood, most artists, musicians and creators have little or no access to the monetary benefits of their own creations.
\item[-]internet service providers such as AT\&T, Comcast and verizon slow down or block any content, applications or websites you may use
\item[-]Most video \& live systems are centralized subject to censorship
\iffalse %direwolff edit
, leading to citizens unwilling or unable to share their free speech openly
\fi
\item[-]As the systems to deploy video and live are expensive and centralized there has been a significant lack of innovation
\iffalse %direwolff edit
\item[-]Monetization systems are also very rudimentary with annoying intrusive advertisements that are hard to avoid during video experiences
\fi
\end{itemize}


We believe a more egalitarian \& democratic open GPUCoin marketplace is the panacea to many of the problems the old systems fail to address:
\begin{itemize}
\item[+]By giving creators a way to monetize their own creations on Hoot, we empower them to create, trusting the open decentralized marketplace to fairly compensate them over archaic centralized controlled channels.
\item[+]\textbf{Net Neutrality}: we strongly support and believe in net neutrality and by design theGPUCoin decentralized net prevents any form of throttling, blocking or slowing down content
% \iffalse %direwolff edit
\item[+]By making the video system decentralized and censorship resistant, GPUCoin IPCN promotes citizen free speech without risk of detection and censorship
% \fi
\item[+]By making video and compute systems more expressive we plan to make video systems democratic and censorship resistant, leading to an unleashing of video innovation on the open decentralized web.

\item[+]By making the platform opensource and decentralized over the blockchain, creators can price their content and accept cryptocurrencies in place of intrusive advertisements which only benefit centralized systems such as Facebook, making this a more seamless experience for consumers. Since cryptocurrencies do not need to be issued by a central authority, financial means of control to censor content is made irrelevant. Blocking a payment channel is a common means of censorship e.g., the Paypal banking account belonging to Wikileaks was blocked by various central controlling organizations.
\end{itemize}

% \fxerror{need to combine all problems or delete $2/3$}
% \fxerror{Replace this section with developer and service provider problems rather than consumer problems - direwolff}


\subsection{Enterprise Customer Problem}
Due to large, distributed teams, combined with complex and highly regulated environments, there is pressure on organizations to support better communication, collaboration, and knowledge capture. With rapid innovation in online and mobile video formats and in video delivery technologies, enterprises that wait years between major platform updates will frustrate customers and employees with outdated experiences. The current live-streaming products are too complex and expensive, leaving customers with cobbled-together solutions that don’t work well and require continuous IT support. The current providers are neither mobile-first nor self-serve, while also suffering from significant latency, high bandwidth, and battery consumption issues. As major browsers no longer support plugins, user experience is fractured. Example enterprise customers are Zoom, Livestream, Twitch, Salesforce, Coursera, Udacity, Stanford University among others.

\subsection{Enterprise Solution / Product Offering}
Hoot will dramatically reduce the pain of running an enterprise scale live-video stream by offering a next generation, powerful platform that: 
\begin{itemize}
\item[-]Streamlines the communication and collaboration between teams, management, partners and customers
\item[-]Gathers, curates and federates remote institutional and field knowledge
\item[-]Provides full live-streaming capabilities without having to know scaling 
\item[-]On-boards new employees, train and support remote field workers, channel partners and customers
\end{itemize}


\subsection{Consumer Customer Business Problem}
The consumer world is ready for mobile live broadcasting. Participation in a live-stream is the next big wave and future of interactive live TV. Facebook serves about 8 Billion video views a day and Snapchat about 10 Billion. Current mobile live-streaming apps do not deliver true real-time, failing in typical real world mobile use cases where high bandwidth and battery usage is unacceptable. Consumers also lose their memorable moments as the live-streams are ephemeral. eSports such as \emph{Dota2} are very popular among gamers and live-streaming audiences alike. The e-sporting events constantly draw audiences in the millions comparable to live football and baseball sporting events. Being able to stream to millions of viewers with near zero latency and HD quality has been a huge challenge for the existing CDN based live-streaming systems. Example consumer customers are Dota2 eSports players, free-lance journalists and social-media celebrities who are looking to build their online presence.

\subsection{Solution to Consumer customer problem / Product Offering}
A consumer grade true real-time live-streaming service needs to be built from the ground up to offer live-streaming of mobile games and eSports, for iPhone, iPad, Android smartphones and tablets. Hoot's break through open source live-streaming technology brings true real-time video in a scalable way to its audience, with just the network latency. Hoot smart mobile streaming, being self adaptive based on network conditions and available bandwidth, results in significantly lower bandwidth and battery consumption, leading to superior user experience. Hoot also allows users to stream from their Mac/PC devices, in addition to the mobile apps, directly to engage their social audience. Hoot is the best way to watch, broadcast interactive live-stream videos and discover talented broadcasters.

% \section{Mission}
% Hoot Mission
% \fxerror*{need to enter Hoot mission or delete}
%
% \section{Vision}
% Hoot Vision
% \fxerror*{need to enter vision or delete}

\subsection{History of Live-Streaming}
Over the last half a century, people have been fascinated by live video. On September 4, 1951, Harry Truman spoke at the Japanese Peace Treaty Conference in San Francisco and, this was the worlds first live broadcast. Four months later, The Today Show would become the first broadcast news program airing live in the United States. Since then, we have loved live video, from our favorite news stations to our guilty pleasures of reality television.

\iffalse
\sout{Streaming is a lot older in its origins than one might intuitively suppose. One of the earliest streaming platforms was Muzak. This along with similar audio systems played continuous music. When we think of streaming, though, we think computers \& the Internet. The full development of that capacity was more recent. Many technical advances in the 1990s \& 2000s improved the bandwidth of networks. This increased the number of people \& computers with access to those networks, creating the Internet as we know it today. Standard formats were also developed \& protocols that we use to code online material \& functions (TCP/IP, HTTP, HTML, etc.).}
\fi

As the bandwidth of connections to the Internet \& computing power available to the average person continued to increase, it was natural that the audio streaming used by Internet radio would graduate to streaming video. Data compression methods contributed a lot to this development as well. Video files contain a lot of information. Compression allows that information to be efficiently transmitted \& stored.

\subsection{Live-Streaming versus On-Demand Streaming}
The term “live-streaming” is sometimes applied where it doesn’t belong. Streaming from a recorded source, which is what one finds on YouTube, Netflix, \& many other commercial streaming sources, is on-demand streaming. This means that the user can watch the content at will, while live-streaming occurs only at the moment, in real-time. Live streaming comes from a content source such as video cameras and microphones. It is made available at the same time as the event being filmed occurs. On-demand streaming provides content from a recorded source instead. Streaming radio \& much Internet television consists of live-streaming. As far as the “streaming” portion of the process is concerned, live \& on-demand streaming are similar from the viewer’s perspective. They are quite different in technical \& procedural details from the standpoint of the producer or broadcaster, though. The main difference from a technical end is the use of temporary storage for the material in progressive streaming or on-demand streaming. This is where a file is partially downloaded, stored to memory, \& played while the next portion of the file is downloading. True streaming or live-streaming does not employ partial memory capture. It streams directly from the source to the user via a computer processor that finalizes the broadcast.


\section{Technical Problem \& solution}

\subsection{Current Centralized Streaming Solution}

\begin{figure}[h!]
 \centering
 % \includegraphics[width=1.0\textwidth]{static/problem-architecture-trans}
 \includegraphics[width=1.0\textwidth]{static/problem-architecture-trans-cmrfont}
 \caption{Current closed, centralized, expensive, censorable live-streaming system}
 \label{image:problem-architecture-trans-cmrfont}
\end{figure}

Figure \ref{image:problem-architecture-trans-cmrfont} shows the state of current live-streaming system. There are 4 components to a live-streaming system. They are explained below.
\subsubsection{Broadcasting Software}
A proprietary mobile video encoding software is used. The primary purpose of this software is to capture video frames and audio, from mobile, desktop, or stand alone cameras. The software encode the captured video and audio frames into a video standard, and a closed/proprietary video streaming protocol and is published to Streaming software.

\subsubsection{Broadcasting Server Software}
Current software that solves this problem are Wowza\footnote{https://www.wowza.com/products} and Adobe\footnote{https://www.adobe.com/products/catalog.html}. The Broadcasting software receives the encoded live streams and generates small fragments of video files that are then published to a Content Delivery Network.

\subsubsection{Centralized Content Delivery Network}
The Broadcasting Server Software publishes the generated video fragment files to Content Delivery Network such as Amazon Cloud-front \footnote{https://aws.amazon.com/cloudfront/} or Microsoft Azure \footnote{https://azure.microsoft.com/en-us/services/media-services/}

\subsubsection{Video Player}
This is the final step in live video streaming. Media/Video/Audio clients for platforms: mobile, desktop, play out the video files from the content delivery networks.


These are the drawbacks with current live-streaming system:
\begin{itemize}
 \item[-]Centralized points of failures: Backend streaming software, relying on content delivery networks
 \item[-]Proprietary and closed source software
 \item[-]Prone to censorship as easy to control/shutdown the service
 \item[-]Expensive licensing fees
\end{itemize}

\subsection{GPUCoin Solution}
 % \fxerror{fix name for everything Hoot GPUCoin network, GPUCoin token, Hoot GPUCoin protocol team, GPUCoin Foundation need right names for all and replace all }.

\begin{figure}[h!]
 \centering
 % \includegraphics[width=1.0\textwidth]{static/gpctokens-solution-trans-ai}
 \includegraphics[width=1.0\textwidth]{static/gpctokens-solution-trans-ai-cmrfont}
 \caption{Open Source, decentralized, GPUCoin IPCN mining, distributed computing system}
 \label{image:gpctokens-solution-trans-ai-cmrfont}
\end{figure}


Figure \ref{image:problem-architecture-trans-cmrfont} is the proposed live-streaming system. They are explained below.
\subsubsection{GPUCoin Open Source Broadcasting Software powered by Real-Time Satoshi Streaming Protocol}
GPUCoin Open Source Broadcasting software captures video frames and audio. All major platforms: iOS, Android, Mac, Windows, Linux will be supported. The captured video frames and audio data are encoded to widely accepted open video and audio formats: H.264 and AAC. The encoded H.264 and AAC audio is published to the GPUCoin Network with \textbf{R}eal-\textbf{T}ime \textbf{S}atoshi \textbf{S}treaming \textbf{P}rotocol(RTSSP).

\subsubsection{GPUCoin Peer-to-Peer mining Node}
Miners in the GPUCoin GCN mining network Peer-to-Peer node will run GPUCoin software sharing unused bandwidth and earning GPUCoins for doing so. This is a highly resilient fault tolerant network, and miners will be able to join or leave the network anytime. The GPUCoin P2P Node replaces the need for a broadcasting server software and expensive content delivery network.

\subsubsection{GPUCoin Open Source Video player}
GPUCoin Open Source Video player plays the live stream in realtime. All major platforms: iOS, Android, Mac, Windows, Linux will be supported.

\subsubsection{Hoot archived videos}
Broadcasted video is continuously archived, and will be stored in decentralized file system IPFS. \footnote{Decentralized IPFS File system https://ipfs.io/} Hence GPUCoin live stream protocol makes the live-streaming faster, safer, and more open by
\begin{itemize}
 \item[+]having no single centralized points of failures - byzantine fault tolerant p2p network
 \item[+]Highly resilient fault tolerant network with Peer-to-Peer nodes
 \item[+]Censorship resistant as there are no centralized points of control
 \item[+]core Peer-to-Peer and networking software is open source
\end{itemize}

\section{What makes Hoot special}
When compared to other products on the market, Hoot has several defensible advantages:
\begin{itemize}
\item[*]Optimized for 2G/3G networks around the world, low CPU and GPU usage, saves battery and bandwidth consumption
\item[*]Next-gen live-streaming product that enables mobile phone self-serve streaming 
\item[*]Allows to interleave background music in a seamless manner
\item[*]Breakthrough patent-pending technology designed from the ground up requiring no licensing fees
\item[*]Hoot offers instant archival of live-stream videos making it unnecessary to upload files again at the end of live-stream
% \iffalse %direwolff edit
\item[*]Modular architecture allows building Tor/VPN modules in order to enable censorship resistant live-streaming to promote free speech
% \fi
\end{itemize}


\section{Traction \& Usage}
A mobile consumer application is currently live on the iOS AppStore and Google Play-store. Hoot has received streams from 

% \fxerror*{put right numbers}

Tables \ref{table:1} and \ref{table:2} show the usage statistics.

\setlength{\arrayrulewidth}{.7mm}
\setlength{\tabcolsep}{18pt}
\renewcommand{\arraystretch}{2.0} 
% \newcolumntype{s}{>{\columncolor[HTML]{AAACED}} p{3cm}}
% \arrayrulecolor[HTML]{DB5800}
 


\begin{table}[!htb]
\centering
\begin{tabular}{ |c|c| }
\hline
\rowcolor{lightgray} \multicolumn{2}{|c|}{User Statistics} \\
% \hline
% Metric Name & Metric Value \\
\hline
Monthly Active Users (MAU) & 214,769 \\
% \rowcolor{gray}
% Total Users & 3,000,000 \\
\hline
\end{tabular}
\caption{User Statistics}
\label{table:1}
\end{table}

\begin{table}[!htb]
\centering
\begin{tabular}{ |c|c| }
\hline
\rowcolor{lightgray} \multicolumn{2}{|c|}{Live Video Streaming Statistics} \\
% \hline
% Metric Name & Metric Value \\
\hline
Number of Videos & 48,207 \\
Average Viewers per stream & 155 \\
Average Stream Duration & 4 minutes 35 seconds \\
Total Time Watched & 37k days or 100 years \\
% \rowcolor{gray}
\hline
\end{tabular}
\caption{Live Video Streaming Statistics}
\label{table:2}
\end{table}


\iffalse
\section{Low Latency Streaming Technology}
\fxerror{maybe not needed}
How Hoot powers low latency streaming.
\fi

\section{Hoot Architecture}

\includegraphics[scale=0.5]{static/hoot-video-architecture-channel-trans}

\subsection{Broadcast Side - mobile iOS client}
The protocol for realtime livestreaming video is called Real-Time Satoshi Streaming Protocol[\textbf{RTSSP}].
Video frames are captured at a resolution of 540x960 to 720x1280 based on network connectivity. Audio stream is captured using the built in iOS device microphone at a sampling rate of 44.1 KHz. Optionally, real time filters (Black and White, Glow, Fisheye, Sepia) can be applied to captured video frames in real-time. Video and audio are encoded using the native hardware H.264(H.265 in android) and AAC encoders, respectively. The video frames are encoded using a VBR algorithm with a maximum bitrate of 1 Mbps, this can be increased for usecases such as VR streaming. Audio stream is encoded in AAC format with a bitrate of 128 Kbps. The H.264 + AAC stream is encoded into an RTSSP stream and is transmitted to Hoot RTSSP server.

\subsection{Broadcast Side — Desktop Mac client}
Video frames are captured at native screen resolution, and audio stream is captured using the built in microphone at a sampling rate of 44.1 KHz. Hoot native cocoa Mac app written in Objective-C supports capturing FaceTime, Screenshare, and a combination of FaceTime and Screenshare. Video frames and audio stream are encoded using the native H.264 and AAC encoders, respectively. The video frames are encoded with a VBR algorithm. Audio stream is encoded in AAC format with a bitrate of 128 Kbps. The H.264 + AAC stream is encoded into an RTSSP stream and is transmitted to the open source RTSSP server.

\subsection{Viewer Side mobile iOS/Android client }
 Hoot open source Native mobile media player decodes RTSSP + H.264 and AAC data to make the live broadcast available to viewer in real-time. The HLS (HTTP Live Streaming) stream that is made available can be played using the iOS/Android Native media players, when the Hoot RTSSP player or app is not available.

\subsection{Viewer Side Mac/ Destkop PC client} 
The RTSSP stream is played using Adobe Flash technology supported by modern browsers. The HLS stream can be played using HTML5 player available in modern browsers.

\subsection{Server Side peer to peer decentralized Technology}
Similar to Bitcoin blockchain technology, any node can join or leave the Hoot network at anytime. Each node runs a realtime broadcasting server.
The GPUCoin network has several RTSSP servers that serve to bootstrap the Network. We use commodity servers with modern processors and with 1 Gbps duplex ethernet; specialized servers are not needed. The hoot server generates two variants of streams: a RTSSP stream and a HLS stream in order to make them accessible in browsers across Windows, Mac OS, Linux and Android platforms. A server with 1 Gbps duplex ethernet can support up to a total of 1000 viewers. A stream is replicated horizontally across multiple servers (without additional latency) to stream to virtually an unlimited number of simultaneous viewers. 

Streamed videos are instantly archived [\emph{H.264+AAC, mp4 container}] in the cloud for later viewing. The archived videos are indexed (scrubbable and quick to scan). We have access to datacenters in the following geographically distributed locations through RTSSP servers to provide the least latency to viewers globally: Amsterdam Netherlands, Frankfurt Germany, Hong Kong, London UK, Melbourne Australia, Queretaro Mexico, Milan Italy, Montreal Canada, Toronto Canada, Paris France, Singapore, Sydney Australia, Tokyo Japan, Dallas TX, Houston TX, San Jose CA, Seattle WA, Washington DC. 
% \sout{}
Streams are replicated and pulled to the closest node to the viewers location, i.e., a viewer in Tokyo Japan viewing a stream from Washington DC would be connected to a replicated stream on the Tokyo Japan hoot node in order to reduce latency.


%inlcude faster than input - input imports all commands http://i.imgur.com/YsPQMeD.png include{filename} gets you the speed bonus, but it also can't be nested, can't appear in the preamble, and forces page breaks around the included text by using \ clearpage before and after the content of the file.. 



\section{Security}
The live connection is encrypted using AES\_256\_CBC, with HMAC-SHA1 for message authentication and DHE\_RSA as the key exchange mechanism. Every Hoot opensource player connection is authenticated.
An authorization key is needed to view a private Hoot video stream. Signup, interactions, HLS streams and archived static content are end-to-end HTTPS SSL encrypted to ensure strong security. 

% \iffalse %direwolff edit
\subsection{Anonymity and privacy over VPN and Tor}
Anonymity and privacy are key to enable free speech, and this matters more in countries where free speech continues to be an ongoing human rights issue. In combination with blockchain technology, the network is
designed to route video streams and meta data over VPN and optionally Tor network to avoid censorship and promote free speech.
% \fi

\section{GPUCoin Monetizing Engine}
GPUCoin tokens based on crypto-currency technology power the GPUCoin marketplace and economy. GPC miners earn GPUCoin tokens running their own open source decentralized Peer-to-Peer GPUCoin nodes utilizing the unused networking bandwidth and compute capacity they may have. 
% \iffalse %direwolff edit
In countries where censorship is an issue they may run decentralized GPC nodes with Tor/VPN modules enabled so they can support free speech through Hoot live-streaming.
% \fi
 GPUCoin tokens can also be used by viewers to support their favorite artists, musicians and gamers. They may send GPUCoin tokens to the streamers they love watching and for events that they want to support. Streamers can also earn GPUCoin tokens by enabling subscriptions in order to have a dependable source of recurring revenue. This enables them to make a living off their fan base from the comfort of where they are at their best without having to spend for event space and the complicated off-line co-ordinating schemes needed to assemble all their fan base for their events.

For micro-payments, artists can safely accept the GPUCoin payment immediately. The size of the payment is too small for the effort to steal it. Micro-payments are almost always for intellectual property, where there is no physical loss to the merchant.

 GPUCoin IPCN network will also build marketing and sales tool to help streamers and gamers market their events and build a paid subscriber base using email lists and SMS lists among other social media channels. 
Musicians can also use the album selling tools to list and sell their albums, singles and release music videos. They can choose to exchange their GPUCoin tokens earned for crypto-currencies or fiat currencies.
 Streamers can also use GPUCoin tokens to purchase advertising space to feature events or utilize the marketing and sales tools to drive more viewers to their streaming events such as an album launch, book launch, movie launch or e-sports gaming event. GPC miners, streamers and viewers can also load GPUCoin tokens on to their respective accounts using crypto-currencies such as Bitcoin, Ethereum, Litecoin, Monero, Zcash and fiat currencies such as USD, EUR among others.


\subsection{MerkleDB: GPUCoin's censorship resistant peer-to-peer identity, trust \& reputation block-chain}
Since there is an economy of trading in the marketplace of the GPUCoin network, having a Peer-to-Peer identity \& reputation database \textsc{MerkleDB} to enable seamless, non-custodial decentralized, trust-free interactions becomes essential. Users/Agents may be identified using Civic, UPort, or using what the Decentralized Identity Foundation\footnote{http://identity.foundation} is building. Feedback \& reviews as well as point scoring out of a maximum of 5 \& minimum of 1 for quality of interactions factor into an agents reputation trust score. The trust score of each agent is hashed into the block-chain using their public GPG key \& hashed user-name or decentralized identity so as to make them censorship resistant. Trust score \& reviews may only be added by anyone to the database \& nothing can ever be removed making this the trust reputation blockchain. \emph{Merkle trees}, an efficient verifiable data-structure, is used to ensure the reputation database is usable by making it possible to download the relevant sub-tree for a particular user's sub-network hash even as the reputation blockchain grows very large in size as the GPUCoin IPCN network MerkleDB grows exponentially in size as the default reputation block-chain database.
The blockchain currently cannot in its current avatars store Equifax type of reputation database in a decentralized fashion, as there are no decentralized trustless anonymized peer to peer reputation databases, where anyone may add trust/credit/metadata scores but no one may remove data, but GPUCoin has a superior decentralized crypto solution to the Equifax hacks with our own GPC Trust \& Reputation DB \textbf{MerkleDB} to the Equifax problem designed using the best practices of cryptography \& computer science fundamentals including GPG public/private key anonymity, Merkle trees to efficiently download/query trust/credit scores, \& mathematically ensure integrity of encrypted anonymized data including identity and their trust/credit scores.

We address this widespread Equifax\footnote{http://fortune.com/2017/10/01/blockchain-equifax/} hack concern as GPUCoin has a market place component \& we need a reliable censorship resistant identity \& reputation database to support reliable credit worthiness and trust. We believe that centralized systems such as Equifax are long obsolete \& need to be replaced by multiple decentralized trust/reputation databases, one, hopefully the most important among them being the MerkleDB, the trust \& reputation database for the decentralize crypto-block-chain web. 

But since MerkledDB is designed to be open, where anyone may democratically add but no one may remove, all identities and trust scores are hashed/anonymized using GPG public keys and only if you permit [via your anonymous private key stored encrypted in your private device] a third party you authorize to view your scores may they download and view your scores. We make the ability to download only your scores efficient, as the GPC Trust \& Reputation DB MerkleDB grows large in size[imagine GBs in size \& millions of users in identity count], using Merkle trees a solid verifiable data-structure that is one of the strong under-pinnings of the Bitcoin block-chain SPV clients. Hence a wide ranging breach like Equifax would be mathematically impossible in the MerkleDB, GPUCoin trust \& reputation database.
And since all identities are anonymized using GPG public keys they have the added benefit of being censorship resistant as well. We plan to build our own GPUCoin P2P trust \& reputation database for supporting GPUCoin market place economics \& commerce initially but once its perfected \& gains traction, we will open source it to other block-chain networks as well which we hope will be another valuable foundational layer contribution as crypto reputation/trust db layer will support more efficient crypto commerce \& economics. We believe by providing such foundational fat protocols including the GPUCoin distributed compute layer and complementary GPC trust/reputation database MerkleDB layer we will substantially increase the overall GDP of the crypto-currency market.


\subsection{Multi-sig escrow wallets}
GPUCoin tokens are first sent to a multi-sig escrow wallet, that is controlled by the buyer/viewer, seller/streamer and an independent 3$^r$$^d$-party escrow. Any two out of the three parties need to sign in order for the transaction to be completed. Also the number of times the buyer or seller necessitates escrow agents to mediate a dispute and the time to complete a transaction will factor into the reputation of the buyer and seller. Any trusted agent with a high enough reputation score can register to be an independent 3$^r$$^d$ party escrow agent. Escrow agents also earn feedback and trust which are hashed and stored in the block-chain using their public GPG key and hashed user-name so it becomes censorship resistant.

\subsection{Vickrey Auction to find optimal price}
To bootstrap, GPUCoin IPCN network will run Vickrey auctions to find the best service to run on the miners computer that has excess GPU capacity. A \emph{Vickrey auction} is one in which the winner pays the second-highest price, not the price they themselves bid, which has been effectively used by Google Adsense and Adwords.
GPUCoin can instantiate any auction protocol, if they find a suitable auction protocol that is superior to Vickrey. Its seed protocol specifies a procedure for stakeholders to approve amendments to the auction protocol, \emph{including} amendments to the auction amendment procedure itself. Upgrades to GPUCoin auction protocol are staged through a testing environment to allow stake-holders and token-holders to recall potentially inferior amendments, that lead to sub-optimal pricing for network stakeholders. 

Since the GPUCoin IPCN network may also be used for other tasks than streaming live video, the network can be extended to run any distributed computing task such as computer graphics, business applications, machine learning, cryptography, malware prevention analysis, science and services, making the GPUCoin IPCN network a Uber for computers, enabling miners to rent their unused CPU/GPU cycles \& get paid in GPUCoin crypto-currency. Hence the GPUCoin decentralized network powers true distributed cloud computing accelerated by GPUs.


\section{GPUCoin video platform, plugins and video AppStore to support machine learning, augmented reality, VR and video v-apps}
We are quite excited by ARKit and ARCore which are going to be soon available in the upcoming versions of iOS11 and Android respectively. We believe this framework is going to usher a golden era of augmenting reality in video-streaming. With Tensor-flow, Caffe, Keras, Theano, TensorFlow Object Detection api, GCP cloud video intelligence API and Azure video machine-learning apis there is going to be a big wave of machine learning video content-analysis applications, makes videos searchable, and discoverable. You can now search every moment of every video file in your catalog and find every occurrence as well as its significance. IPCN GPC video primitives via api/SDK will allow developers to extract actionable insights from video files without requiring any machine learning or computer vision knowledge. 

We envision an Augmented Reality first OS, an operating system native to virtual and augmented reality. We want to enable a thriving ecosystem by building a video appstore and provide a scalable platform for video developers around the world. By enabling a platform with easily extensible and scriptable plugins, and video appstore for AR and ML/AI v-apps, we will accelerate the golden age of intelligent live video and augmented reality. By providing an extensible plugin architecture, we will allow developers to build plugins on the IPCN video architecture. Filters, face-detection and face-swapping are some early augmented reality ideas that will be explored.
\iffalse
\sout{}
\fi
Some interesting machine-learning and AI ideas are Label Detection(detect entities within the video, such as "dog", "flower" or "car"), Shot Change Detection (detect scene changes within the video), Regionalization (automagically specify a region where processing will take place), automated subtitle detection, home/office security \& other applications yet to be discovered may also be encouraged on the GPUCoin distributed computing platform.

The underlying GPUCoin technology may also be used to build an open source low cost security \& surveillance alternative to closed systems such as Nest.
% \iffalse %direwolff edit
\subsection{Hoot Augmented Reality ads - Performant AR ads with measurable ROI - Building Adwords of video }

Video Ads have been priced historically based on cost per million impressions(CPM) model for impressions served as it has been nearly impossible to know how much of these video ads lead to a product sale or if they even positively improve the ad-spend ROI for the advertisers topline. Viewers do not know what action to take \& how leaving them confused to figure out how to follow up on the video ad they just saw, leading to viewer engagement drop off leading to lower sales \& \underline{poor ad performance}. Hence it has \underline{not} been easy to measure ROI, effectiveness \& performance of video ads historically.

Using interactive Hoot AR augmented reality ads with obvious call to actions i.e., buy button or rent button below the interactive Hoot AR ad, sales can be generated immediately after the viewer interacts with the Hoot AR ad. Payments or micro-payments are collected using crypto-currencies such as bitcoin, eth, litecoin \& monero. Hoot is now able to price the ads based on sales referred from the user interaction with AR ad(referrals). So instead of ROI blind pricing based on impressions(CPM) we can price \underline{ROI aware} based on \underline{cost per referral(CPR)}. The CPR price becomes the primary signal that drives the continuous Vickrey auction engine that powers the Hoot ad marketplace underlineasizing the importance of ad-spend ROI and value of referrals to the advertisers. Since \underline{CPR video ads perform 10x better than CPM ads} they do not need to be shown as frequently as CPM ads to generate the same revenue, hence showing more relevant ads with higher conversions less frequently than CPM video ads leads to a \underline{quantifiably superior user experience}. By bringing about an innovative CPR based business model to video ads using Augmented reality \underline{call-to-actions} we bring the effectiveness of Google AdWords to video ads that thus far only performs as poorly as CPM banner ads. Google Adwords \& Amazon affiliate program proved \underline{conversions} \& \underline{referrals} are more valuable than impressions in keyword advertising \& e-commerce respectively. We want to bring this accountability to the video ad world by ushering in CPR based performance pricing model. Hence Hoot AR ads with AR call to actions \& interactivity, improves the effectiveness of plain video ads, leading to a quantum improvement in video ad-spend ROI very much like AdWords improved the ROI of web based banner ads, laying the foundation for a multi-billion dollar ads business. 

% \fi

% \iffalse


\section{the GPUCoin Foundation and Governance} % (fold)
\label{sec:the_foundation_and_governance}
% As a company limited by guarantee established in Switzerland, t
The GPUCoin Foundation's primary objective is to promote the real world application of the GPUCoin Decentralized Open Live-Streaming platform. It also aims to initially develop the GPUCoin platform and advocate governance and transparency for the platform. The GPUCoin Foundation will establish an association consisting of members of the GPUCoin ecosystem, which will be empowered to determine the direction of functionality and improvement to the GPUCoin Distributed GPU accelerate computing, mining platform and associated ecosystem.
% section the_foundation_and_governance (end)

\subsection{The dispute resolution process} % (fold)
\label{sub:the_dispute_resolution_process}
The GPUCoin Foundation will specify a dispute resolution process, utilizing an internationally accepted dispute resolution system. A rotating board of dispute referees will monitor disputes through the resolution process, \& oversee collateral release to plaintiffs. Note that this board of dispute referees is not the dispute resolution process specifically; rather it is the mechanism through which dispute resolutions can be enacted through the release of collateral on the block-chain.


\subsection{GPUCoin Token sales} % (fold)
\label{sub:hoot_token_sales}
The GPUCoin Foundation will fund the development of the GPUCoin IPCN Platform discussed in this paper through the issuance of GPUCoin tokens. These tokens will run natively on the Ethereum block-chain and will be offered to backers of the GPUCoin IPCN project via a token sale.
% The token sale will be launched on or about the November 21 2017.
A second token sale will take place once the initial prototype has been developed and tested to fund its deployment. 
\iffalse
For more information on the GPUCoin token, see \_\_\_ \fxerror{add correct section}.
\fi

\subsection{Token Allocation \& Distribution} % (fold)
\label{sub:token_allocation_and_distribution}
 The supply of GPUCoin token is limited to the number of one hundred million (100,000,000) in total (including those available for sale during the Token Sale) \& will be generated upon the launch ("Token Launch".)

 The tokens will be distributed in the following manner:
80\% (30/30/20) of the tokens will be eventually allocated amongst the community; the remaining 20\% will be allocated to the GPUCoin Foundation initiator, early backers, \& the GPUCoin protocol network development team.


\begin{table}[!htb]
\centering
\begin{tabular}{ |p{2.8cm}|p{2.5cm}|p{5cm}|}
\hline
\rowcolor{lightgray} \multicolumn{3}{|c|}{GPUCoin Token Distribution Model} \\
\hline
Channels & Percentage & Lock up Period \\
\hline
30,000,000 GPUCoin token Sale (GTS) & 30\% &
% Token Sale - Launch 21 November 2017.
The initial funding will be used to develop a working prototype, financial setup, legal fees \& promotion.
 \\
 \hline
30,000,000 Additional GPUCoin Token Sale (AGTS) & 30\% & Additional GPUCoin Token Sale. On the release of a successful prototype, a second token sale will be launched to fund the full production ready launch \& development of all relevant technology \& organization matters.
\\
\hline
20,000,000 GPUCoin tokens Retained by the GPUCoin Foundation as Treasury & 20\% & 100\% of which locked for 24 months.Strategic Planning, Project Support, Token Swap, Emergency Fund, Development \& Legal Fees - These will be subject to a 2 year lock-up. Subsequent to the lock-up, these will be used for various development \& operation costs of GPUCoin Platform over 2 further years.
\\
\hline
20,000,000 GPUCoin Advisors, Directors and Early Backers & 20\% & 70\% of which is locked for 12 months. 30\% of which is locked up for 24 months. Distributed to the directors, advisors, and early backers of the
 project.
\\
 \hline
\end{tabular}
\caption{GPUCoin Token Distribution Model}
\label{table:hoot_token_distribution_model}
\end{table}
\iffalse
\fxerror{clean up table, better formatting/fonts}
\fi

\subsection{Restriction on the use of the funds} % (fold)
\label{sub:restriction_on_the_use_of_the_funds}
To remain in line with the spirit of the project’s open and transparent philosophy, all funds shall be tracked and reported according to the GPUCoin Foundation’s guidelines. A custodian will monitor the usage of the digital tokens and share it with the community periodically.

\begin{enumerate}
 \item Financial planning and reporting
 \begin{itemize}
 \item The GPUCoin Foundation shall develop financial planning and review financial performance of the previous quarter.
 \end{itemize}

 \item Digital tokens management
 \begin{itemize}
 \item The digital tokens belonging to the GPUCoin Foundation shall be managed by authorized personnel. The security of digital tokens is ensured by multi signature technology.
 \end{itemize}

 \item Digital wallet protocol
 \begin{itemize}
 \item The GPUCoin Foundation’s digital wallet shall be protected by a multiple signature technology mechanism.
 \end{itemize}

 \item Disclosure
 \begin{itemize}
 \item On a regular basis, the GPUCoin Foundation shall disclose on the topics regarding community matters, including status of development, operations, and the usage of tokens, as well as whether the GPUCoin Foundation operates in accordance with the governance policy.
 \end{itemize}
\end{enumerate}

% \fi

\section{GPUCoin Development Time-line}
The Hoot consumer mobile app which uses Facebook or Twitter to authenticate is already live in the iTunes AppStore\footnote{Hoot live on iOS AppStore https://appsto.re/us/40RS-.i} and Google Android Play Store\footnote{Hoot Live on Google Playstore https://play.google.com/store/apps/details?id=com.onhoot.android}.
A light weight performant native mac app is live on
the website \footnote{Download link for Hoot Live on Mac Desktop https://onhoot.com/mac}. The mac app can be used to screen-share meetings, conferences and webinars. It can also be used to live-stream desktop games such as Minecraft, league of legends, world of warcraft and others.

A native enterprise version that uses Slack for authentication of internal private teams is already live.
 This requires quite a bit of work to integrate with the slack teams API and also in order ensure security for private teams. Following platforms are supported
\begin{itemize}

\item[-]iOS app for slack private teams \footnote{ iOS private Hoot business client for slack teams http://hootvideo.com/business}
\item[-]Hoot Mac desktop app for slack private teams \footnote{Desktop Hoot client for Slack teams http://hootvideo.com/macbusiness}
\item[-]All modern browsers. \footnote{Slack based private team build of Hoot https://hootvideo.com}
\end{itemize}

Web browser end points are live on line as well
\footnote{Hoot live link on Web browser https://onhoot.com}. The minimum requirements are any modern
browser such as Safari, Mozilla Firefox, Microsoft Internet Explorer or Google Chrome which fall-back to HTML5 HLS video format for playback
of the live-streams.
% \iffalse %direwolff edit
\subsection{Tor and VPN to enable censorship resistant live-streaming }
Tor modules to live-stream video over the onion routed tor network needs to be built. Integration with VPN needs to be built in order to evade censorship. This would enable true zero knowledge live-streams and computing in countries where censorships and free speech continue to be ongoing human rights issues.
% \fi

\subsection{Focus on Performance}
We have a strong focus on performance and highly performant applications while still maintaining smaller binary sizes and code integrity. The Hoot iOS app is under 10MB, the latency is under a second plus the network latency. This leads to a superior user experience and efficient usage of unused compute.

\section{Uber for GPU accelerated computers creating IPCN - Inter-Planetary Compute Network}
GPUCoin IPCN network is a dense Byzantine fault tolerant peer-to-peer network - creating the worlds first IPCN - Interplanetary compute network. GPUCoin IPCN Network is based on a complex architecture revolving around Peer-to-Peer, Block-chain, Smart Contracts, State Channels. GPUCoin IPCN network protocol will enable the creation of decentralized compute network, powered by decentralized crypto-currency micro-payments. This leads to Uber for computers helping create the worlds first IPCN an interplanetary compute network. We will create a platform to create new compute primitives using any Turing compute programming language. We will use container technologies such as docker and kubernetes to efficiently distribute and use excess unused compute. The compute results of the network are verifiable using cryptographic and mathematical properties of the cryptographic design. The IPCN takes advantage of the coming Cambrian explosion of computing, crypto-currencies \& CPU/GPU miners. While Nvidia is like Tesla and Intel is like GM/Ford, GPUCoin IPCN is like Uber's network but for GPU accelerated computers, helping the network participants i.e., miners turn their null GPU/CPU cycles into valuable crypto-currency GPUCoins GPCs.


\section{Conclusion}
We are building the layer that will power decentralized trusted secure computation. We have an opportunity, arguably a generational opportunity for reinventing computation thereby fostering a new era for trust, legitimacy \& decentralization. We believe several next generation Ð-applications will be built on the \textbf{I}nter-\textbf{P}lanetary \textbf{C}ompute \textbf{N}etwork powered and incentivized by GPUCoin crypto-currency tokens. We believe GPUCoin will provide a fertile platform for decentralized cloud distributed computing in trust-less environments.


Bet on the future with IPCN GPUCoin mining, distributed computing protocol using GPUCoin for mining and decentralized distributed computing.



% \renewcommand{\lstlistingname}{Appendix}
% \begin{lstlisting}[caption={Digital Fingerprint},captionpos=b, language=java,numbers=none]

% {
% "$schema": "digital_fingerprint",
% "definitions": {},
% "id": "https://hootvideo.com/whitepaper",
% "properties": {
% "compressedContent": {
% "id": "/properties/compressedContent",
% "items": {
% "id": "/properties/compressedContent/items",
% "type": "integer"
% },
% "type": "array"
% },
% "link": {
% "id": "/properties/link",
% "type": "string"
% },
% "name": {
% "id": "/properties/name",
% "type": "string"
% },
% "publishDate": {
% "id": "/properties/publishDate",
% "type": "string"
% }
% },
% "type": "object"
% }

% \end{lstlisting}
\newpage
\listoffigures
\newpage 
\listoftables
\newpage 
\bibliographystyle{plain}
\end{document}

\end{lstlisting}

\bibliographystyle{plain}
\end{document}

