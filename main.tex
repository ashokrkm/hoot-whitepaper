\documentclass{article}
\usepackage[utf8]{inputenc}
\usepackage{listings}
\usepackage{float}
\title{Hoot de-centralized censorship free open source Live Streaming Protocol and Marketplace Technical Whitepaper}
\author{Hoot Author}
\date{July 26 2017}
\setlength{\parskip}{1em}
\usepackage{natbib}
\usepackage{graphicx}
\usepackage{amssymb}
\usepackage{amsmath}
\usepackage[normalem]{ulem}
\usepackage{soul}
 
\begin{document}

\maketitle

\begin{abstract}
The Hoot team is building the next generation technology for live
Streaming services  based on blockchain technology and a new
innovative de-centralized live streaming technology that would
completely eliminate expensive content delivery networks and use de-centralized peer-to-peer networks for delivering live streaming video.


\st{We will use crypto-financing (Initial Coin Offering) for capital rather than traditional venture capital and shareholders.}

\end{abstract}
\newpage

\tableofcontents
\newpage

\section{Introduction}
Introduction for Hoot


\section{Mission}
Hoot Mission

\section{Vision}
Hoot Vision

\section{ERC223 Compatibility}
We are monitoring the ERC223 token standard proposal\footnote{https://github.com/ethereum/EIPs/issues/223} and are factoring future compatibility of into the design of our Namespace Hoot tokens.

\section{What makes Hoot special}
Hoot special

\subsection{Problem \& solution}
Unfettered, censorship free live streaming

\section{Traction \& Usage}
table/ graph of usage with GB etc

\section{Low Latency Streaming Technology}
How hoot powers low latency streaming.

\section{Anonymity and privacy over VPN and Tor}
Anonymity and privacy are key to enable free speech, and this matters
more so in countries where free speech continues to be an ongoing
issue. In combination with blockchain technology, the network is
designed to route video streams and meta data over VPN and optionally
Tor network to evade censorship.

\section{Hoot Monetizing Engine}
Hoot tokens based on crypto-currency technology power the hoot
marketplace and economy. Hoot miners earn hoot tokens running their own open source
de-centralized hoot nodes utilizing the unused networking bandwidth
and compute capacity they may have. In countries where censorship is an issue they
may run de-centralized hoot nodes with Tor/VPN modules enabled so they can
support free speech through hoot
live-streaming. Hoot tokens can also be used by viewers to support their favorite artists,
musicians and gamers. They may send hoot tokens to the
streamers they love watching and for events that they want to
support. Streamers can also earn hoot tokens by enabling subscriptions in order to have a
dependable source of recurring revenue. This enables them to make a
living off their fan base from the comfort of where they are without
having to spend for event space and the complicated offline
co-ordinating schemes needed to assemble all their fan base for their events.

 Hoot network will also build marketing and sales tool to help
streamers and gamers market their 
events and build a paid subscriber base using email lists and sms lists among other social media
channels. 
Musicians can also use the album selling tools to list and sell
their albums, singles and release music videos. They can
choose to exchange their hoot tokens earned for crypto-currencies or fiat currencies.
 Streamers can also use hoot tokens to
purchase advertising space to feature events or utilize the marketing and sales
tools to drive more viewers to their
streaming events such as an album launch, book launch, movie launch or
e-sports gaming event. Hoot miners, streamers and viewers can also load Hoot
tokens on to their respective accounts using crypto-currencies such as Bitcoin,
Ethereum, Litecoin, Monero, Zcash and fiat currencies such as USD, EUR among others.

\section{Un-censorable P2P identity and reputation database}
Since there is an economy of trading in the marketplace of the Hoot network, having a Peer to peer
identity and reputation database to enable seamless, non-custodial
de-centralized, trust-free interactions becomes essential. Feedback and
reviews as well as point scoring out of a maximum of 5 and minimum of
1 for quality of interactions factor into an agents reputation trust
score. The trust score of each agent is hashed into the blockchain
using their public gpg key and hashed username so as to make them Un-censorable.

\section{Multi-sig escrow wallets}
Hoot tokens are first sent to a multi-sig escrow wallet, that is controlled by the buyer/viewer, seller/streamer and an
independent third-party escrow. Any
two out of the three parties need to sign in order for the transaction to be
completed. Also the number of times the buyer or seller necessitates
escrow agents to mediate a dispute and the time to complete a
transaction will factor into the reputation of the buyer and
seller. Any trusted agent with a high enough reputation score can
register to be an independent third party escrow agent. Escrow agents
also earn feedback and trust which are hashed and stored in the
blockchain using their public gpg key and hashed username so it
becomes un-censorable.

\section{Hoot Development Timeline}
The hoot consumer app which uses facebook or twitter to authenticate
is already in the iTunes appstore\footnote{https://appsto.re/us/40RS-.i} and android play-store\footnote{https://play.google.com/store/apps/details?id=com.onhoot.android}.
There is a light weight performant native  mac app which is live on
the website \footnote{http://onhoot.com/mac}. The mac app can be used
to screen-share meetings, conferences and webinars. It can also be used
to livestream desktop games such as Minecraft, league of legends,
world of warcraft and others.

Web browser end points are live on line as well
\footnote{http://onhoot.com}. The minimum requirements are any modern
browser such as Safari, Mozilla Firefox, Microsoft Internet Explorer
or Google Chrome with fallback to HTML5 HLS video format for playback
of the live-streams.

A native enterprise version that uses Slack for authentication of
internal private teams is already live.\footnote{http://hootvideo.com}

Tor modules to live-stream video over the onion routed tor network needs
to be built. Integration with VPN needs to be built in order to evade censorship. This would enable true zero knowledge live-streams in
countries where censorships and free speech continue to be ongoing
human rights issues.

The underlying hoot technology may also be used to build an open
source low cost security and surveillance alternative to closed systems
such as Nest.

\section{Conclusion}
Bet on the future with Hoot live streaming protocol.

\newpage

% \renewcommand{\lstlistingname}{Appendix}
% \begin{lstlisting}[caption={Digital Fingerprint},captionpos=b, language=java,numbers=none]

% {
%     "$schema": "digital_fingerprint",
%     "definitions": {},
%     "id": "https://hootvideo.com/whitepaper",
%     "properties": {
%         "compressedContent": {
%             "id": "/properties/compressedContent",
%             "items": {
%                 "id": "/properties/compressedContent/items",
%                 "type": "integer"
%             },
%             "type": "array"
%         },
%         "link": {
%             "id": "/properties/link",
%             "type": "string"
%         },
%         "name": {
%             "id": "/properties/name",
%             "type": "string"
%         },
%         "publishDate": {
%             "id": "/properties/publishDate",
%             "type": "string"
%         }
%     },
%     "type": "object"
% }

% \end{lstlisting}

\bibliographystyle{plain}
\end{document}

