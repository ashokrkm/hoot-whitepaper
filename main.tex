% compiling and viewing latex in os x
% brew cask install mactex
% /Library/TeX/texbin/pdflatex main.tex 

\documentclass{article}
\usepackage[utf8]{inputenc}
\usepackage{listings}
\usepackage{float}
\title{Hoot de-centralized censorship free open source Live Streaming Protocol and Marketplace Technical Whitepaper}
\author{Hoot Author}
\date{July 26 2017}
\setlength{\parskip}{1em}
\usepackage{natbib}
\usepackage{graphicx}
\usepackage{amssymb}
\usepackage{amsmath}
\usepackage[normalem]{ulem}
% \usepackage{soul}
 
\begin{document}

\maketitle

\begin{abstract}
The Hoot team is building the next generation technology for live
Streaming services based on blockchain technology and a new
innovative open source de-centralized live streaming protocol that would completely eliminate expensive content delivery networks and use peer-to-peer networks for delivering not just live streaming video but also archived videos.

\sout{We will use crypto-financing (Initial Coin Offering) for capital rather than traditional venture capital and shareholders.}

\end{abstract}
\newpage

\tableofcontents
\newpage

\section{Introduction}
\subsection{History of Live Streaming}
Over the last half a century or so, people of world have been fascinated by live video. On September 4, 1951, Harry Truman spoke at the Japanese Peace Treaty Conference in San Francisco and, for the first time ever, it was broadcasted live. Four months later, The Today Show would become the first broadcast morning news program airing live in the U.S. Since then, we’ve loved live video, from our favorite news stations to our guilty pleasures of reality television.

Streaming is a lot older in its origins than one might intuitively suppose. One of the earliest streaming platforms was Muzak\footnote{https://en.wikipedia.org/wiki/Muzak}. This along with similar audio systems played continuous music. When we think of streaming, though, we think computers and the Internet. The full development of that capacity was more recent. Many technical advances in the 1990s and 2000s improved the bandwidth of networks. This increased the number of people and computers with access to those networks, creating the Internet as we know it today. Standard formats were also developed and protocols that we use to code online material and functions (TCP/IP, HTTP, HTML, etc.).

As the bandwidth of connections to the Internet and computing power available to the average person continued to increase, it was natural that the audio streaming used by Internet radio would graduate to streaming video. Data compression methods contributed a lot to this development as well. Video files contain a lot of information. Compression allows that information to be efficiently transmitted and stored.


\section{Blockstack, Token, Status and the beginnings of decentralized
  web}
 Firefox, Chrome and IE have ruled the centralized web. De-centralized technologies
such as Blockstack are ushering in the auspicious beginnings of
de-centralized web. 
 We are now entering a new era of De-Centralized applications, blockchain technologies collectively known as Web 3.0. In the centralized web, the users are the product,
their interests, preferences are sliced and diced by companies such as
Facebook, Twitter and Google and sold to advertisers, enriching their
small group of shareholders driven by profit, with significant barriers to entry. In the
decentralized web, the user information is private. Decentralized
technologies such as Status, Token and Hoot empower the network token
holders, who can have various motivations other than mere profit,
including privacy, altruism and a more inclusive distribution of
control and information.


\subsection{Live Streaming versus On-Demand Streaming}
The term “live streaming” is sometimes applied where it doesn’t belong. Streaming from a recorded source, which is what one finds on YouTube, Netflix, and many other commercial streaming sources, is on-demand streaming. This means that the user can watch the content at will, while live streaming occurs only at the moment, in real-time. Live streaming comes from a content source such as video cameras and microphones. It is made available at the same time as the event being filmed occurs. On-demand streaming provides content from a recorded source instead. Streaming radio and much Internet television consists of live streaming.
As far as the “streaming” portion of the process is concerned, live and on-demand streaming are similar from the viewer’s perspective. They are quite different in technical and procedural details from the standpoint of the producer or broadcaster, though. The main difference from a technical end is the use of temporary storage for the material in progressive streaming or on-demand streaming. This is where a file is partially downloaded, stored to memory, and played while the next portion of the file is downloading. True streaming or live streaming doesn’t employ partial memory capture. It streams directly from the source to the user via a computer processor that finalizes the broadcast.




\section{Mission}
Hoot Mission

\section{Vision}
Hoot Vision

\section{ERC223 Compatibility}
We are monitoring the ERC223 token standard proposal\footnote{https://github.com/ethereum/EIPs/issues/223} and are factoring future compatibility into the design of our Namespace Hoot tokens.

\section{What makes Hoot special}
Hoot special

\subsection{Problem \& solution}
Unfettered, censorship free live streaming

\section{Traction \& Usage}
table/ graph of usage with GB etc

\section{Low Latency Streaming Technology}
How hoot powers low latency streaming.

\section{Hoot Architecture}

\includegraphics[scale=0.5]{static/hoot-video-architecture-channel-trans}

\subsection{Broadcast Side - mobile iOS client}
The protocol for realtime livestreaming video is called Real-Time Satoshi Streaming Protocol[\textbf{RTSSP}].
Video frames are captured at a resolution of 540x960 to 720x1280 based on network connectivity. Audio stream is captured using the built in iOS device microphone at a sampling rate of 44.1 KHz. Optionally, real time filters (Black and White, Glow, Fisheye, Sepia) can be applied to captured video frames in real-time. Video and audio are encoded using the native hardware H.264(H.265 in android) and AAC encoders, respectively. The video frames are encoded using a VBR algorithm with a maximum bitrate of 1 Mbps, this can be increased for usecases such as VR streaming. Audio stream is encoded in AAC format with a bitrate of 128 Kbps. The H.264 + AAC stream is encoded into an RTSSP stream and is transmitted to Hoot RTSSP server.

\subsection{Broadcast Side — Desktop Mac client}
Video frames are captured at native screen resolution, and audio stream is captured using the built in microphone at a sampling rate of 44.1 KHz. Hoot native cocoa Mac app written in Objective-C supports capturing FaceTime, Screenshare, and a combination of FaceTime and Screenshare. Video frames and audio stream are encoded using the native H.264 and AAC encoders, respectively. The video frames are encoded with a VBR algorithm. Audio stream is encoded in AAC format with a bitrate of 128 Kbps. The H.264 + AAC stream is encoded into an RTSSP stream and is transmitted to the open source RTSSP server.

\subsection{Viewer Side mobile iOS/Android client }
 Hoot open source Native mobile media player decodes RTSSP + H.264 and AAC data to make the live broadcast available to viewer in real-time. The HLS (HTTP Live Streaming) stream that is made available can be played using the iOS/Android Native media players, when the Hoot RTSSP player or app is not available.

\subsection{Viewer Side Mac/ Destkop PC client} 
The RTSSP stream is played using Adobe Flash technology supported by modern browsers. The HLS stream can be played using HTML5 player available in modern browsers.

\subsection{Server Side peer to peer decentralized Technology}
Similar to Bitcoin blockchain technology, any node can join or leave the Hoot network at anytime. Each node runs a realtime broadcasting server.
The GPUCoin network has several RTSSP servers that serve to bootstrap the Network. We use commodity servers with modern processors and with 1 Gbps duplex ethernet; specialized servers are not needed. The hoot server generates two variants of streams: a RTSSP stream and a HLS stream in order to make them accessible in browsers across Windows, Mac OS, Linux and Android platforms. A server with 1 Gbps duplex ethernet can support up to a total of 1000 viewers. A stream is replicated horizontally across multiple servers (without additional latency) to stream to virtually an unlimited number of simultaneous viewers. 

Streamed videos are instantly archived [\emph{H.264+AAC, mp4 container}] in the cloud for later viewing. The archived videos are indexed (scrubbable and quick to scan). We have access to datacenters in the following geographically distributed locations through RTSSP servers to provide the least latency to viewers globally: Amsterdam Netherlands, Frankfurt Germany, Hong Kong, London UK, Melbourne Australia, Queretaro Mexico, Milan Italy, Montreal Canada, Toronto Canada, Paris France, Singapore, Sydney Australia, Tokyo Japan, Dallas TX, Houston TX, San Jose CA, Seattle WA, Washington DC. 
% \sout{}
Streams are replicated and pulled to the closest node to the viewers location, i.e., a viewer in Tokyo Japan viewing a stream from Washington DC would be connected to a replicated stream on the Tokyo Japan hoot node in order to reduce latency.



\section{Security}
The live connection is encrypted using AES\_256\_CBC, with HMAC-SHA1 for message authentication and DHE\_RSA as the key exchange mechanism. Every Hoot opensource player connection is authenticated.
An authorization key is needed to view a private Hoot video stream. Signup, interactions, HLS streams and archived static content are end-to-end HTTPS  SSL encrypted to ensure strong security.    

\section{Anonymity and privacy over VPN and Tor}
Anonymity and privacy are key to enable free speech, and this matters
more so in countries where free speech continues to be an ongoing
issue. In combination with blockchain technology, the network is
designed to route video streams and meta data over VPN and optionally
Tor network to evade censorship.

\section{Hoot Monetizing Engine}
Hoot tokens based on crypto-currency technology power the Hoot
marketplace and economy. Hoot miners earn hoot tokens running their own open source
de-centralized hoot nodes utilizing the unused networking bandwidth
and compute capacity they may have. In countries where censorship is an issue they
may run de-centralized hoot nodes with Tor/VPN modules enabled so they can
support free speech through hoot
live-streaming. Hoot tokens can also be used by viewers to support their favorite artists,
musicians and gamers. They may send hoot tokens to the
streamers they love watching and for events that they want to
support. Streamers can also earn hoot tokens by enabling subscriptions in order to have a
dependable source of recurring revenue. This enables them to make a
living off their fan base from the comfort of where they are without
having to spend for event space and the complicated offline
co-ordinating schemes needed to assemble all their fan base for their events.

 Hoot network will also build marketing and sales tool to help
streamers and gamers market their 
events and build a paid subscriber base using email lists and sms lists among other social media
channels. 
Musicians can also use the album selling tools to list and sell
their albums, singles and release music videos. They can
choose to exchange their hoot tokens earned for crypto-currencies or fiat currencies.
 Streamers can also use hoot tokens to
purchase advertising space to feature events or utilize the marketing and sales
tools to drive more viewers to their
streaming events such as an album launch, book launch, movie launch or
e-sports gaming event. Hoot miners, streamers and viewers can also load Hoot
tokens on to their respective accounts using crypto-currencies such as Bitcoin,
Ethereum, Litecoin, Monero, Zcash and fiat currencies such as USD, EUR among others.

\section{Auction and finding best price for unused compute}


\section{Un-censorable P2P identity and reputation database}
Since there is an economy of trading in the marketplace of the Hoot network, having a Peer to peer
identity and reputation database to enable seamless, non-custodial
de-centralized, trust-free interactions becomes essential. Feedback and reviews as well as point scoring out of a maximum of 5 and minimum of 1 for quality of interactions factor into an agents reputation trust score. The trust score of each agent is hashed into the blockchain using their public gpg key and hashed username so as to make them Un-censorable.

\section{Multi-sig escrow wallets}
Hoot tokens are first sent to a multi-sig escrow wallet, that is controlled by the buyer/viewer, seller/streamer and an
independent third-party escrow. Any
two out of the three parties need to sign in order for the transaction to be
completed. Also the number of times the buyer or seller necessitates
escrow agents to mediate a dispute and the time to complete a
transaction will factor into the reputation of the buyer and
seller. Any trusted agent with a high enough reputation score can
register to be an independent third party escrow agent. Escrow agents
also earn feedback and trust which are hashed and stored in the
blockchain using their public gpg key and hashed username so it
becomes un-censorable.

\section{Hoot Development Timeline}
The Hoot consumer mobile app which uses Facebook or Twitter to authenticate is already live in the iTunes AppStore\footnote{Hoot live on iOS AppStore https://appsto.re/us/40RS-.i} and Google Android Play Store\footnote{Hoot Live on Google Playstore https://play.google.com/store/apps/details?id=com.onhoot.android}.
A light weight performant native  mac app is live on
the website \footnote{Download link for Hoot Live on Mac Desktop https://onhoot.com/mac}. The mac app can be used to screen-share meetings, conferences and webinars. It can also be used
to livestream desktop games such as Minecraft, league of legends,
world of warcraft and others.

Web browser end points are live on line as well
\footnote{Hoot live link on Web browswer https://onhoot.com}. The minimum requirements are any modern
browser such as Safari, Mozilla Firefox, Microsoft Internet Explorer
or Google Chrome which fallback to HTML5 HLS video format for playback
of the live-streams.

A native enterprise version that uses Slack for authentication of
internal private teams is already live.\footnote{Slack based private team build of Hoot https://hootvideo.com}

Tor modules to live-stream video over the onion routed tor network needs
to be built. Integration with VPN needs to be built in order to evade censorship. This would enable true zero knowledge live-streams in
countries where censorships and free speech continue to be ongoing
human rights issues.

The underlying hoot technology may also be used to build an open
source low cost security and surveillance alternative to closed systems
such as Nest.

\section{Hoot carbon footprint, mining, scarcity and profitability}
In late 2013, 8.25 megatonnes (8,250,000 tonnes) of CO$_2$ per year
was estimated to be the carbon footprint of Bitcoin per year\footnote{https://pando.com/2013/12/16/bitcoin-has-a-dark-side-its-carbon-footprint/}. These
computers are consuming so much electricity that it’s already
unprofitable to mine in some regions of the world. Since excess
bandwidth and compute capacity is utilized towards streaming,
encoding, object recognition
and security of video and audio streams the resources otherwise would
be utilized are profitably used.  Since Hoot tokens are fairly distributed to miners corresponding to their compute and bandwidth availability irrespective of how much CPU they control, the Hoot carbon footprint will be exponentially lower than the Bitcoin network which depends on continuously increasing complexity of the hashing required for mining. Since hoot tokens can only be mined or acquired from the platform, they will tend to be a scarce token.

\section{Uber for Computers}
Since the hoot network can also be used for other tasks that streaming live video, the network can be extended to run any computing task such as computer graphics, business applications, machine learning, crytography, malware prevention analysis, science and services, making the Hoot network a Uber for computers, enabling miners to rent their unused CPU/GPU cycles and get paid in Hoot cryptocurrency. Hence the Hoot decentralized network powers true cloud computing.

\section{Currency And Issuance}

The Hoot network includes its own built-in currency, HOOT Coins, which serves the dual purpose of providing a primary liquidity layer to allow for efficient exchange between various types of digital assets and, more importantly, of providing a mechanism for paying transaction fees.

The issuance model will be as follows:

\begin{itemize}

\item HOOT Coins will be released in a currency sale at the price of 1000-2000 HOOT Coins per BTC, a mechanism intended to fund the Hoot organization and pay for development that has been used with success by other platforms such as Mastercoin, ETH, Tezos and NXT. Earlier buyers will benefit from larger discounts. The BTC, ETH, XMR, and LTC received from the sale will be used entirely to pay salaries and bounties to developers and invested into various for-profit and non-profit projects in the Hoot cryptocurrency ecosystem.

\end{itemize}


\section{Conclusion}
Bet on the future with Hoot live streaming protocol.

\newpage

% \renewcommand{\lstlistingname}{Appendix}
% \begin{lstlisting}[caption={Digital Fingerprint},captionpos=b, language=java,numbers=none]

% {
%     "$schema": "digital_fingerprint",
%     "definitions": {},
%     "id": "https://hootvideo.com/whitepaper",
%     "properties": {
%         "compressedContent": {
%             "id": "/properties/compressedContent",
%             "items": {
%                 "id": "/properties/compressedContent/items",
%                 "type": "integer"
%             },
%             "type": "array"
%         },
%         "link": {
%             "id": "/properties/link",
%             "type": "string"
%         },
%         "name": {
%             "id": "/properties/name",
%             "type": "string"
%         },
%         "publishDate": {
%             "id": "/properties/publishDate",
%             "type": "string"
%         }
%     },
%     "type": "object"
% }

% \end{lstlisting}

\bibliographystyle{plain}
\end{document}

