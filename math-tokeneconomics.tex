\subsection{GPUCoin math and token economics}
Running an AWS p2.xlarge, the lowest priced GPU unit, costs about \emph{awsdc} = \$0.9/hr. This unit is equipped with a Tesla K80 GPU, along with 64GB RAM and a 4 core CPU. 
Hence the dollar cost of an average AWS\footnote{as AWS and GCP have comparable costs} GPU instance per hour at about 1\$ per hour is a fair assumption \footnote{ varies between \~ 0.9 - 1.0 \$ in both Google compute platform and Amazon web services}

The \textbf{monthly dollar cost of AWS GPU instance} is

\emph{p} = \emph{awsdc}*24*30 = 1*24*30 = \textbf{720\$  } [1\$ per hour per aws GPU instance * 24 hours per day * 30 days per month]

Ethereum mining revenue from one AWS GPU instance

\emph{e} = 31\$

The GPUCoin instance cost is pegged at 1/5th AWS costs to stay competitive and as GPUCoin nodes do not pay infrastructure and electricity costs in the decentralized model. This can be generalized as \emph{peg} variable that GPUCoin miners can tweak based on the availability and scarcity of GPUs and render farms in their region.

\emph{peg} = 0.5

The dollar cost for a GPUCoin instance comparable to an AWS gpu instance 

\emph{gpcdc} = \emph{peg}*\emph{awsdc} =  1/5* AWS cost = 1/5 * 1\$ = 0.2\$.

Hence 0.2\$ is the dollar cost per comparable GPUCoin instance per hour as the GPUCoin IPCN network does not have to pay for infrastructure and electricity, these are sunk costs for the GPUCoin network miners.

The \textbf{monthly dollar cost of a comparable GPUCoin instance} is 

\emph{q} = \emph{gpcdc}*24*30 = 0.2*24*30 = \textbf{144\$ }

[ 0.2\$ per hour per GPU instance * 24 hours per day * 30 days per month ].


Taking \emph{t} = 20\$ for transaction fees and margins

GPUCoin mining revenue per month for 1 GPUCoin instance
\emph{r} = \emph{q} - \emph{t} = 144-20 = 124\$

\emph{r} = 4* \emph{e} 

Hence GPUCoin mining is 4x more profitable than ethereum mining which is a profitable gpu mined cryptocurrency. This is a reasonable profit for something that takes no effort, and comparable to Bitcoin mining profits during the golden GPU mining era circa 2009-2011.

Initial GPUCoin token difficulty for mining follows the bitcoin mining model i.e., the token max limit is a hard cap of \emph{c} = 21 million GPUCoins to honor satoshi nakamoto’s invaluable bitcoin contribution to the fellow crypto brethren. Early decentralized GPUCoin nodes earn more GPUCoins as a block reward as there are fewer nodes in the network; the proof of compute and proof of streaming difficulty increases with the number of GPUCoin nodes in the network.

* whereas earlier \emph{1000 GPUCoins} would be needed to buy time in \emph{1 single GPUCoin instance}
* later as the network strengthens and more nodes join the GPUCoin network 1 single GPUCoin would be able to buy time in \emph{1000+ GPUCoin nodes} few years down the line as GPUCoin tokens appreciate in value due to scarcity of mining, a hard cap of \emph{c} = \textbf{21 million} GPUCoins and strong sustainable demand for GPUCoin instances that will be generated by the significantly cheaper than AWS pricing of comparable GPU instance.

When there are just 10 GPUCoin nodes in the network the PoC and proof of streaming difficulty is roughly 1 hour of streaming needed to earn 1 GPUCoin as block reward. As more GPUCoin nodes join the GPUCoin network the PoS difficulty goes up, when there are 10000 GPUCoin nodes in the network the mining difficulty is about 1000 hours of streaming needed to earn 1 GPUCoin as block reward.

Irrespective of the GPUCoin price and mining difficulty the GPUCoin instance dollars costs are pegged at 1/5th of the AWS instance dollar costs. This ensures that the GPUCoin dollar costs stay competitive and attractive compared to the AWS instance dollar costs in order to drive sustained GPUCoin instance demand.

As mining difficulty is adjusted based on the number of the nodes compute capacity that the network controls and the maximum number of GPUCoins that can ever be mined is hard-capped at 21 million GPUCoins the GPUCoin cryptocurrency is designed to be a deflationary currency relative to fiat currencies similar to bitcoin and litecoin.

While the miner who finds the winning block earns the GPUCoin block reward all the competing GPUCoin streaming nodes earn the transaction fees so in Bitcoin there is one winner and losers in GPUCoin there is one winner and no losers or multiple miners sharing mining rewards corresponding to their \emph{Proof of Stake} or the number of GPUCoin nodes they used for PoC or PoS mining.
